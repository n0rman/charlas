%Esta charla puede ser descargada en https://github.com/n0rman/charlas/blob/master/IntegracionSLEdu/IntegracionSLEdu.tex
%Autor: Norman GArcía Aguilar norman@debian.org.ni

\documentclass{beamer}
%\usepackage[latin1]{inputenc}
\usepackage[spanish]{babel}
\usepackage{graphicx}
\usetheme{Warsaw}
\title{Integraci\'on de las TICs basadas en Software Libre en los sistemas educativos de Nicaragua}
\author[n0rman]{Norman Garc\'ia \\ \texttt{norman@debian.org.ni}}
\institute{Debian Nicaragua}
\date{Noviembre 7, 2012}
%\showboxdepth=5
%\showboxbreadth=5
\begin{document}


\begin{frame}
	\titlepage
\end{frame}

\begin{frame}{Contenido}
	\tableofcontents
\end{frame}


\section{Introducci\'on}

\begin{frame}
\frametitle{?`Qu\'e es Software Libre?}
        
	\begin{itemize}
                \pause \item Libertad \alert{0} de ejecutar el programa con cualquier prop\'osito.
		\pause \item Libertad \alert{1} de estudiar como funciona el programa y modificarlo.
		\pause \item Libertad \alert{2} de redistribuir copias.
		\pause \item Libertad \alert{3} de distribuir copias de versiones modificadas.
       \end{itemize}

\end{frame}

\begin{frame}
\frametitle{El Software Libre en Nicaragua}    
        \begin{itemize}
                \pause \item Nace en \alert{1990} con NicaLUG.
		\pause \item En \alert{2006} nace RSLCAN y SeLibre.
		\pause \item En \alert{2007} se forman comunidades Ubuntu, Debian, Fedora, OpenSuse Nicaragua. GUL-NIC.
	     	\pause \item GUL-NIC inicia LinuxTour, FLISoL, SFD, DFD, ECSL.
		\pause \item Se inician proyectos de colaboraci\'on como \alert{Guardabarranco} y \alert{OSM-NI}.
    \end{itemize}
\end{frame}


\begin{frame}
\frametitle{Infraestructura tecnol\'ogica en las escuelas p\'ublicas}
	\begin{enumerate}
		\pause \item 621 escuelas p\'ublicas cuenta con laboratorios de computaci\'on.
		\pause \item Sistema de apadrinamiento.
		\pause \item 12 padrinos identificados.
		\pause \item Dos modalidades identificadas:
			\begin{itemize} 
				\item Laboratorio de computadoras.
				\item UCPN.
			\end{itemize}
		\pause \item Si no hay apadrinamiento, laboratorio se convierte en C.T.E.
	\end{enumerate}
%        \begin{center}
%                 \includegraphics[scale=0.50]{../img/}
%        \end{center}
\end{frame}

\begin{frame}
\frametitle{Software Libre en la Educaci\'on}
             \begin{itemize}
		\pause \item Permite cumplir con sus misiones fundamentales: \alert{difundir} el conocimiento y ense\~nar a los estudiantes a ser \alert{buenos miembros} de su comunidad. 
		\pause \item Software Libre no es solo algo t\'ecnico, es algo \alert{\'etico}, social y pol\'itico.
		\pause \item \alert{Libertad} y cooperaci\'on son valores esenciales del Software Libre.
		\pause \item \alert{Compartir} es bueno y \'util para el progreso de la humanidad.
		\pause \item Licencias de Software Privativo tiene un costo econ\'omico.
	     \end{itemize}

%Economizar: escuelas no tienen bastante dinero, no deben desperdiciar pagando licencias.
%Algunas empresas suelen regalar copias gratuitas, adicción?
%Misión Social: educar como buenos ciudadanos, de una socierda capaz fuerte independiente solidaria y libre
%Educación en ciudadania: No solo enseñar técnicas, también espiritu de buena voluntad y el habito de ayudar al projimo.
\end{frame}

\section{Experiencias}

\begin{frame}
\frametitle{Colegio Guardabarranco}
	\begin{itemize}
		\pause \item En 2005 MINED inicia proyecto piloto con apoyo de cooperaci\'on extreme\~na.
		\pause \item Se lleva a cabo con estudiantes de IE y MINED.
		\pause \item Se realizan capacitaciones en moodle, LinexEDU y Jclic.
		\pause \item 2 a\~nos despu\'es, docentes TICs deciden volver al uso de Sofware Privativo.
	\end{itemize}
\end{frame}

\begin{frame}
\frametitle{Fundaci\'on Zamora Ter\'an}
	\begin{itemize}
		\pause \item 2009 Nace la Fundaci\'on Zamora Ter\'an. RSE del Grupo Financiero LAFISE.
		\pause \item Basado en el proyecto OLPC. Computadora XO utiliza el sistema operativo Fedora + Sugar.
		\pause \item 30,000 estudiantes. 856 docentes. 105 escuelas, 85 de ellas con Internet.
		\pause \item Creaci\'on de actividades por desarrolladores del pa\'is.
		\pause \item FZT mantiene su compromiso en el uso de SL, como parte de los cincos principios de OLPC, lo que fomenta el esp\'iritu de colaboraci\'on para los ni\~nos y docentes.
	\end{itemize}
\end{frame}	

\begin{frame}
\frametitle{AFT}
	\begin{itemize}
		\pause \item En 2010, Fundaci\'on Telef\'onica inicia el programa Proni\~no AFT en 24 escuelas.
		\pause \item Se decide usar Ubuntu (\alert{Software Libre}) de manera provisional.
		\pause \item Docentes TICs reciben manuales de instalaci\'on y de uso de Ubuntu.
		\pause \item Aulas TICs empiezan migraci\'on a Software Privativo debido a:
			\begin{enumerate}
			\pause \item Falta de capacitaci\'on.
			\pause \item Desconocimiento de aplicaciones equivalentes.
			\pause \item Resistencia de algunos docentes.
			\end{enumerate}
		\pause \item Colegio Jos\'e de la Cruz Mena sigue utilizando Ubuntu.
	\end{itemize}
\end{frame}

\begin{frame}
\frametitle{RedProCom}
	\begin{itemize}
		\pause \item La \alert{Red de Profesores cristianos de Computaci\'on} nace en 2010.
		\pause \item Siguiendo la filosof\'ia y \'etica cristiana, deciden usar Software Libre, Lubuntu, Ubuntu, Debian como S.O y tecnolog\'ia LTSP.
		\pause \item Actualmente formada por 16 escuelas ubicadas en el pa\'is y con reuniones mensuales.
		\pause \item Laboratorios con hardware muy variado, desde 8 a 20 computadoras.
		\pause \item Estudiantes se autoimponen retos para aprender un S.O que no conoc\'ian. Capta mucho su atenci\'on.
		\pause \item Aprovechamiento de material audiovisual.
		\pause \item Uso de aplicaciones como GIMP, KTouch, GCompris.
		\pause \item Iniciando proceso de transversalizaci\'on.
	\end{itemize}	
\end{frame}

\section {Recomendaciones y conclusiones}
\begin{frame}
\frametitle{Conclusiones}
	\begin{enumerate}	
		\pause \item Instituciones de gobiernos y organizaciones interesadas en el uso de Software Libre.
		\pause \item Resistencia por parte de actores involucrados.
		\pause \item Actores con poco conocimiento en el uso de Software Libre.
		\pause \item Falta de motivaci\'on a los docentes para transversalizaci\'on de las TICs.
		\pause \item Existe un aislamiento de parte de las iniciativas.
                \pause \item \alert{Resistencia} y \alert{desconocimiento} como barreras.
	\end{enumerate}
\end{frame}


\begin{frame} 
\frametitle{Recomendaciones}
	\begin{enumerate}
		\pause \item Propiciar la creaci\'on de un grupo Software Libre y educaci\'on.
		\pause \item Realizar capacitaciones en el uso de Software Libre.
		\pause \item Identificar aplicaciones de Software Libre que se adecuen a la curr\'icula educativa.
		\pause \item Realizar capacitaciones para la creaci\'on de recursos educativos.
		\pause \item Crear un repositorio central de recursos educativos.
	\end{enumerate}
\end{frame}


\section{Gracias}


\begin{frame}
\frametitle{FIN}
	\begin{enumerate}
		\pause \item Presentaci\'on: Integraci\'on de las TICs basadas en Software Libre en Nicaragua.
		\pause \item Presentado por: Norman Garc\'ia  \texttt{norman@debian.org.n}
		\pause \item Noviembre 7, 2013
		\pause \item Licencia: Creative Commons Atribuci\'on-CompartirIgual 3.0 Unported (CC BY-SA 3.0)
	\end{enumerate}

	\begin{center}
  		 \includegraphics[scale=0.20]{../img/cclogo.png}
	\end{center}

\end{frame}
\end{document}
