%Esta charla puede ser descargada en https://github.com/n0rman/charlas/blob/master/HerramientasColaborativas/HerramientasColaborativass.tex
%Autor: Norman GArcía Aguilar norman@debian.org.ni

\documentclass{beamer}
%\usepackage[latin1]{inputenc}
\usepackage[spanish]{babel}
\usepackage{graphicx}
\usetheme{Warsaw}
\title{Herramientas colaborativas para el desarrollo de software}
\author[n0rman]{Norman Garc\'ia \\ \texttt{norman@debian.org.ni}}
\institute{Debian Nicaragua}
\date{Octubre 24, 2015}
%\showboxdepth=5
%\showboxbreadth=5
\begin{document}


\begin{frame}
	\titlepage
\end{frame}

\begin{frame}{Contenido}
	\tableofcontents
\end{frame}


\section{Introducci\'on}

\begin{frame}
\frametitle{?`Qu\'e es una Herramienta de trabajo colaborativo?}
        
	\begin{itemize}
        \pause \item Utilizada para comunicarse entre personas.
		\pause \item Una manera de acortar distancias.
		\pause \item Trabajo en conjunto.
       \end{itemize}

\end{frame}

\begin{frame}
\frametitle{Tipos de herramientas}
	\begin{enumerate}
		\pause \item Herramientas web no gestionadas.
		\begin{itemize}
			\pause \item No tenemos que preocuparnos por implementaci\'on y mantenimiento.
			\pause \item \alert{¡GRATIS!} y muchas veces no libres.
			\pause \item Algunas veces se necesita pagar para tener mejores funcionalidades.
		\end{itemize}
		\pause \item Herramientas web auto gestionadas.
		\begin{itemize}
			\pause \item \alert{¡LIBRES!}
			\pause \item Con todas las funcionalidades completas.
			\pause \item Podemos ajustarlas a nuestras necesidades.
			\pause \item Necesitamos dedicar hardware propio.
		\end{itemize}
    \end{enumerate}
    
\end{frame}
 
\section{Gesti\'on de c\'odigo fuente}

\begin{frame}
\frametitle{Github}    
        \begin{center}
                 \includegraphics[scale=0.50]{../img/github-logo-600x270.png}
        \end{center}
		\begin{itemize}
			\pause \item Plataforma de hospedaje de proyectos.
			\pause \item Muchas funcionalidades: Wiki, Sitio web, est\'adisticas.
		\end{itemize}
\end{frame}

\begin{frame}
\frametitle{Bitbucket}
		\begin{itemize}
			\pause \item Plataforma de hospedaje de proyectos.
			\pause \item Integraci\'on con JIRA.
			\pause \item Wikis
		\end{itemize}
		\begin{center}
                 \includegraphics[scale=0.50]{../img/bitbucket.png}
        \end{center}
\end{frame}


\begin{frame}
\frametitle{Gitosis/Gitweb}
\begin{itemize}
			\pause \item Gitosis es una herramienta para control de acceso a repositorios Git.
			\pause \item Gitweb es una interfaz gr\'afica para visualizar repositorios Git.
			\pause \item Acceso sobre ssh.
			\pause \item Se requiere instalar en un equipo propio.
		\end{itemize}
        \begin{center}
                 \includegraphics[scale=0.50]{../img/gitweb.png}
        \end{center}
\end{frame}

\section {Tracking}

\begin{frame}
\frametitle{Trello}
		\begin{itemize}
			\pause \item Sistema no gestionado.
			\pause \item Aplicaci\'on web para manejo de proyectos.
		\end{itemize}
		\begin{center}
                 \includegraphics[scale=0.50]{../img/trello-calendar.jpg}
        \end{center}
		
\end{frame}

\begin{frame}
\frametitle{Waken}
		\begin{itemize}
			\pause \item Sistema autogestionado.
			\pause \item Aplicaci\'on web para manejo de proyectos (versi\'on \alert{libre} de Trello).
 		\end{itemize}
		\begin{center}
                 \includegraphics[scale=0.50]{../img/wekan.png}
        \end{center}
	
\end{frame}

\begin{frame}
\frametitle{Trac}
		\begin{itemize}
			\pause \item Sistema autogestionado y \alert{libre}.
			\pause \item Integra wiki con gesti\'on de reporte de fallos.
			\pause \item Provee interfaz web para svn y git.
 		\end{itemize}
		\begin{center}
                 \includegraphics[scale=0.50]{../img/orzone_trac_kanban_dashboard.png}
        \end{center}
	
\end{frame}



\section{Documentaci\'on}

\begin{frame}
\frametitle{Wiki}    
		\begin{itemize}
			\pause \item Utilizado para creaci\'on y administraci\'on de documentaci\'on.
			\pause \item Wikispaces, wikia...
			\pause \item ... o una instancia mediawiki en tus servidores.
 		\end{itemize}

\end{frame}


\begin{frame}
\frametitle{Sphinx}    
		\begin{itemize}
			\pause \item Herramienta para creaci\'on de documentaci\'on.
			\pause \item Formatos de s\'alida variadas (HTML, LaTeX, Pub, texto plano). 
			\pause \item Utilizado por proyectos como django, pip, blender, ubuntu.
 		\end{itemize}

\end{frame}


\section{Comunicaci\'on}
\begin{frame}
	\frametitle{As\'incrona}
		\begin{itemize}
			\pause \item Discusi\'on p\'ublica y transparente.
			\pause \item Nos permite tener discusiones t\'ecnicas y mantener a nuestros usuarios informados.
		\end{itemize}
       \begin{center}
                 \includegraphics[scale=0.50]{../img/debianMl.png}
        \end{center}
\end{frame}

\begin{frame}
\frametitle{S\'incrona}
	\begin{itemize}
		\pause \item \alert{I}nternet \alert{R}elay \alert{C}hat
		\pause \item Jit.si
	\end{itemize}
\end{frame}
	

\begin{frame}
	\frametitle{Y muchos más}    
	\begin{itemize}
	\pause \item FusionForgee, Launchpad.
	\pause \item Redmine, Asana, Taiga.io, JIRA.
	\pause \item ?`Qu\'e otras conoc\'es?
	\end{itemize}
\end{frame}

\section{Gracias}


\begin{frame}
\frametitle{FIN}
	\begin{enumerate}
		\item Presentaci\'on: Herramientas colaborativas para el desarrollo de software.
		\item Presentado por: Norman Garc\'ia  \texttt{norman@debian.org.ni}
		\item Octubre 24, 2015
		\item Licencia: Creative Commons Atribuci\'on-CompartirIgual 3.0 Unported (CC BY-SA 3.0)
	\end{enumerate}

	\begin{center}
  		 \includegraphics[scale=0.20]{../img/cclogo.png}
	\end{center}

\end{frame}
\end{document}
